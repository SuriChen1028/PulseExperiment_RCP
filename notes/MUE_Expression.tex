%% LyX 2.3.6.2 created this file.  For more info, see http://www.lyx.org/.
%% Do not edit unless you really know what you are doing.
\documentclass[english]{article}
\usepackage{geometry}
 \geometry{
 a4paper,
 total={170mm,257mm},
 left=20mm,
 top=20mm,
 }
\usepackage[T1]{fontenc}
\usepackage{esint}
\usepackage{babel}
\begin{document}
\title{{\normalsize{}Marginal Utility of Emissions:}}

\maketitle
From the paper, the marginal utility of emissions is given in equation
(11). This equation gives us the following expression:

\[
\frac{\eta}{\tilde{e}}=-\frac{d\phi(y)}{dy}\sum_{\ell=1}^{L}\omega_{\ell}\theta_{\ell}-\frac{d^{2}\phi(y)}{(dy)^{2}}|\varsigma|^{2}\tilde{e}+\frac{(1-\eta)}{\delta}[(\gamma_{1}+\gamma_{2}y)\sum_{\ell=1}^{L}\omega_{\ell}\theta_{\ell}+\gamma_{2}|\varsigma|^{2}\tilde{e}]
\]

The discounted value would be given by 

\[
\exp(-\delta t)\frac{\eta}{\tilde{e}}=\exp(-\delta t)\{-\frac{d\phi(y)}{dy}\sum_{\ell=1}^{L}\omega_{\ell}\theta_{\ell}-\frac{d^{2}\phi(y)}{(dy)^{2}}|\varsigma|^{2}\tilde{e}+\frac{(1-\eta)}{\delta}[(\gamma_{1}+\gamma_{2}y)\sum_{\ell=1}^{L}\omega_{\ell}\theta_{\ell}+\gamma_{2}|\varsigma|^{2}\tilde{e}]\}
\]

We can then integrate that up to get the aggregated marginal utility
impact of reducing emissions by one unit

\[
\int_{0}^{\infty}\exp(-\delta t)\frac{\eta}{\tilde{e}}dt=\int_{0}^{\infty}\exp(-\delta t)\{-\frac{d\phi(y)}{dy}\sum_{\ell=1}^{L}\omega_{\ell}\theta_{\ell}-\frac{d^{2}\phi(y)}{(dy)^{2}}|\varsigma|^{2}\tilde{e}+\frac{(1-\eta)}{\delta}[(\gamma_{1}+\gamma_{2}y)\sum_{\ell=1}^{L}\omega_{\ell}\theta_{\ell}+\gamma_{2}|\varsigma|^{2}\tilde{e}]\}dt
\]

In practive, the emissions pathways only go out 80 years, so we would
get something like

\[
\sum_{t=0}^{80}\exp(-\delta t)\frac{\eta}{\tilde{e}}dt=\sum_{t=0}^{80}\exp(-\delta t)\{-\frac{d\phi(y)}{dy}\sum_{\ell=1}^{L}\omega_{\ell}\theta_{\ell}-\frac{d^{2}\phi(y)}{(dy)^{2}}|\varsigma|^{2}\tilde{e}+\frac{(1-\eta)}{\delta}[(\gamma_{1}+\gamma_{2}y)\sum_{\ell=1}^{L}\omega_{\ell}\theta_{\ell}+\gamma_{2}|\varsigma|^{2}\tilde{e}]\}dt
\]

I think the discrete time approximation for $\exp(-\delta t)$ is
$\frac{1}{1+\delta t}$.
\end{document}
